\documentclass{report}
\usepackage{amsmath}
\usepackage{graphicx}
\usepackage{caption}
\begin{document}
\title{\textbf{Compact Objects and Equation of States}}
\author{Subham Das}
\maketitle 
\begin{center}
\title{PRABHAT KUMAR COLLEGE,CONTAI \\
M.Sc(Physics) 4th Semester -2021 \\
Name- Subham Das \\
Roll-PG/VUEGS37/PHS/IVS    No- 036 \\
REG.No -231667 \\
Session- 2014-2015\\
Under the guidance of :-  Dr. Debasis Atta(Assistant Prof. ,Department of Physics ,Kharagpur College) \\
Subject of Project :- Study of the Compact Objects and their Equation of States \\
Name of the project :- Compact Objects and Equation of States\\
 }
\maketitle
\end{center}
\newpage
\begin{center}
\maketitle{ACKNOWLEDGEMENT}
\paragraph{ }
I express my deep sense of gratitude to my supervisor  Dr. Debasis Atta(Assistant Professor,Kharagpur College) for his kind help , and guidance and advice and precious time and constant encouragement to carry out my project work.
\paragraph{ }
I am also thankful to all the honourable teachers of Department of Physics ,P.K.College,Contai
\paragraph{ }
......................................................
\begin{flushright}
\begin{figure}[hb!]
\includegraphics[width=0.4 \textwidth]{my_signature}\\
Subham Das\\
(PG 4th sem)
\end{figure}
\end{flushright}
\end{center}
\newpage 
\begin{center}
\title{\textit{CERTIFICATE}}
\maketitle
\paragraph{ } 
I am glad to certify that Subham Das , a student of PG 4ht sem in Department of Physics ,P.K.College, contai ,performed a project work named as "Compact Object and Equation of states" under my guidance and supervision.
\paragraph{ }
I believe that he have learned something new out of this project and encouragement of doing something new .
\paragraph{ }
I wish him all the best for his upcoming life.
\paragraph{ }
..................................................................\\
\begin{flushright}
\begin{figure}[hb!]
\includegraphics[width=0.4\textwidth]{signature}\\
..............................................\\
Dr.Debasis Atta\\
Assistant prof.\\
Department of Physics\\
Kharagpur College
\end{figure}
\end{flushright}
\end{center}
\newpage
\chapter{INTRODUCTION}
\subsection{What is compact objects}
Compact objects such as -white dwarfs,neutron starts,and black holes -- are 'born' when a normal star 'die',that is when most of their nuclear fuel has been consumed.

All three species of compact object differ from normal stars in two fundamental ways
 
1)as the stars at the last stage of their stellar evolution,have no nuclear fuel left to burn they can not support themselves stable against their own gravitational pull towards core by generating thermal pressure .Instead the three different compact object deals with this situation in different way ,the WHITE DWARF are stabilize themselves through the pressure of degenerate electron gas,while the NEUTRON stars stabilize themselves predominantly through the degenerate neutron,BLACK HOLES on the other  hand are completely collapsed giant stars that can't find any way of stabilizing against its gravitational pull towards it's core,all the three compact object White dwarf,Neutron stars and Black holes are static over the lifetime of the universe except from some "mini" black holes having mass 10$^{15}$

2.)The 2nd distinguishable characteristic of compact object that differ from normal stars is their exceedingly small size,compared to normal stars ,compact star have much smaller radii therefore resulting a much stronger gravitational fields.Due to the vastly spanned density range of compact object the analysis of a compact object require well understanding of the structure of matter and the nature of the inter-particle force .
\newline
Due to the smaller radii,luminous white dwarfs ,radiate away their residual energy are characterized by much higher effective temperature then normal stars even though they have luminosity lower then a normal stars as we know that for a black body of temperature T and radius R the flux varies as T$^4$ so does the luminosity varies as R$^2$T$^4$ thus white dwarfs are much "whiter" than usual stars thus they are named as "White dwarf"
On the other hand nothing,nor light neither matter can escape the from a black hole,thus an isolated black whole appears as just"black" to any observer and the name Neutron stars derive form the predominance of neutron in them,as of their densities are comparable to the nuclear value ,neutron stars are basically like a "giant nuclei" almost 10$^{57}$ held together by self-gravity.
  white dwarfs can be observed directly through a optical telescopes during their long cooling epoch,and a Neutron stars can be observed directly as a pulsating radio source and indirectly as gas-accrediting,periodic X-ray sources.But in case of observing a Black-hole it can't be observed directly only can be observed through their influence they exerts on their environment. 
\newline

\subsection{Formation of Compact Objects:}
  The compact objects like White dwarf ,neutron starts , black holes are the last stage, end results of the stellar evolution,the factor that determines mainly whether it will end up as a white dwarf or a neutron star,or black hole is thought to be the star's mass 
 White dwarfs are believed to be born from the light star with masses $M\le 4M_\odot$ (where the $M_\odot$ denotes the solar mass) and there is a mass limit called Chandrashakher mass limit which is around$~\sim$ 1.4$M_\odot$.
\newline
  On the other hand neutron stars and black holes are originated from more massive stars,although the line between the stars that form the neutron stars and those that forms black holes are not very uncertain though as the final stages of stellar evolution of massive stars not very  well understood.
\newline
\begin{center}
\centering
{
outcome of stellar evolution and mass
}
\end{center}
\begin{center}
\centering
{
TABLE 1
}
\end{center}
\begin{center}
\begin{tabular}{|c|c|}
\hline 
Range of Mass & Result Expected\\
\hline
\hline
$\le M_\odot$ & lifetime longer than age of universe\\
\hline
$1\le M/M_\odot \le ( 3 to 6)$ & white dwarf +planetary nebula\\

\hline
$(3 to 6)\le  M/M_\odot \le  (5 to 8)$ & degenerate ignition of C$^{12}$+C$^{12}$ \\

& pulsationally driven mass loss to white dwarf\\
\hline
$(5 to 8)\le M/M_\odot \le (60 to 100)$& core collapse +supernova = leads to nutron star\\
&sometimes black hole(?)\\
\hline
$(60 to 100) \le M/M_\odot$ & instability\\
\hline
\end{tabular}
\end{center}
	There is at least two additional black hole formation processes is proposed by theoretical physicist which is only hypothetical all though,the collapse of a hypothetical "super massive star"(of mass almost mass $M \geq$ $8M_\odot$)  which in return result formation of a "super massive black hole" ,after reaching a certain critical density through radiative cooling and contraction,which may result in origin of super massive black holes of mass $M$/$M_\odot$ $~\sim$ 10$^6$-10$^9$
\newline
	The other theory that was proposed for creation of black holes is the existence of primordial black holes in the early universe ,since as per Hawking the "mini black holes" with mass $M \le$ 10$^{15}$ will radiate away their mass through Hawking process in time less then the age of the universe thus as result only primordial black holes with$ M\geq$  10$^{15}$ will remain.
\paragraph{ }
(the whole project is written using the software latex and the graphs are plotted using the python library matplotlib)

\chapter{WHITE DWARFS}
Compact stars such as white dwarfs and neutron stars are the final stage in the evolution of ordinary stars,after a star burned all its hydrogen and helium, the helium in star's core is burned up forming carbon oxygen ,the  nuclear process in the core of star stops and the temperature of the gradually radiates away as a consequence the star stars to shrink increasing its core pressure.
\paragraph{}
As the star started to shrink the pressure in the stars core increase continuously,star having mass above a certain limit, the pressure stars burning of heavier elements causing a new fusion to start.If the star's mass $M<8M_\odot$ they gravitational pressure is not sufficient to reach the pressure resulting in failing to achive the temperature and the critical mass density to start a new fusion of heavier elements.
Due to the very high core temperature the outer layers of the star swells up and get blown away,resulting in compact core which is mainly composed of carbon and oxygen,and the residual star radiates  it thermal energy away,thus forming a white dwarf.
\paragraph{ }
After the heavier elements get burned away and the star cooled down by rediating its heat away there is should be something that stops the star from collapsing by it's gravity,and keeping it in the hydrostatic equilibrium. Due to Pauli principle no two fermion can not occupy the same quantum state,hence the electrons fills up phase space from the lowest energy state with increasing density,as result the remaining electrons sit in physical state with in increasing momentum. And the this resulting high velocity country-balancing the gravity's pull and stabilising the white dwarf
\paragraph{ }
\subsection{\textit{STRUCTURE EQUATION}}
There is generally two type of force that acting on a star,one of which is gravity and other one is the thermal pressure,for compact object like white dwarf the fusion process don't take place so here the thermal pressure is replace by degeneracy pressure of electrons,due to their smaller masses the electrons are the first particle to degenerate; although the white dwarfs can contains other elements such as carbon oxygen but the nuclei contribution to the degeneracy pressure is negligible due to their heavy mass and low momentum compeared to electron.
\paragraph{ }
The pressue and force relation in this case is quite simple,
\begin{center}
\begin{equation}
dp = \frac{dF}{A} = \frac{dF}{4\pi r^2}
\label{1}
\end{equation}
\end{center}
\paragraph{ }
where p is the pressure,F is the force and A denotes the on which the force is acting. Now for gravitational force and due to spherical symmetry of white dwarf we have as ,
\begin{center}
\begin{equation}
dF = -\frac{Gdm\cdot m(r)}{r^2}
\label{2}
\end{equation}
\end{center}
\paragraph{}
where r being the radial distance for the center and dm denotes the elementary mass at distance r. Here mass we considering mass as function of r, such that mass varies as the distance from the centre varies and m(r) denoting mass till radius r ; if $\rho(r)$ being the mass density then we have 
\begin{center}
\begin{equation}
dm = \rho(r)dV = \rho(r)4\pi r^2 dr
\label{3}
\end{equation}
\end{center}
\paragraph{ }
Now substituting the \ref{2} and \ref{3} in the \ref{1} we get the following,
\begin{center}
\begin{equation}
\frac{dp}{dr} = -\frac{G\rho(r)m(r)dr}{r^2}
\label{4}
\end{equation}
\end{center}
\paragraph{ }
here G is the Newton's Gravitational constant and V is the volume then using \ref{3} we have;
\begin{center}
\begin{equation}
\frac{dm}{dr} = \rho(r)4\pi r^2
\label{5}
\end{equation}
\end{center}
\paragraph{ }
we can express the $\rho(r)$ in terms of energy as ;
\begin{center}
\begin{equation}
\rho(r) = \frac{\epsilon(r)}{c^2}
\label{6}
\end{equation}
\end{center}
\paragraph{}
here c is the speed of light and $\epsilon(r)$ denotes the energy density, with these the structure equation of a star( which describes the change of mass and pressure with radius ) can be written as follow;
\begin{center}
\begin{equation}
\frac{dm}{dr} = \frac{4\pi r^2\epsilon(r)}{c^2}
\label{7}
\end{equation}
\end{center}
\begin{center}
\begin{equation}
\frac{dp}{dr} = -\frac{G\epsilon(r)m(r)}{c^2 r^2}
\label{8}
\end{equation}
\end{center} 
\paragraph{ }
The above two equations \ref{7} and \ref{8} the so called structure equation of a star. Now as we can see from the above two equation is that $\frac{dm}{dr}$  is positive and $\frac{dp}{dr}$ is negative which means the mass increase with r while pressure reduce with increasing r ; which is justified as from \ref{7} we can see that dm $\propto$ $r^2$ while from \ref{8} we can see that dp $\propto$ $\frac{1}{r^2}$ 
\paragraph{ }
Now to solve the coupled equation \ref{7} and \ref{8} we need some initial  values, stating with positive values of mass and pressure for some region where mass increase with increasing radius while the pressure decrease with increasing radius eventually getting zero,so we can start with m(r=0)= 0 and some initial pressure as p(r=0) = $p_0$ in order to solve the couple equation 
\subsection{\textit{EQUATION OF STATE}}
\paragraph{}
The phrase "Equation of State" means a thermodynamic equation relating the state veriable such as the condition that describes the state of the matter like pressure, energy,volume which describes the state of the system,in our case the the state equation will a relation between the pressure(p) and energy density ($\epsilon$).From \ref{7} and \ref{8} we can see that both mass and pressure of the white dwarf depends on the energy density $\epsilon(r)$, now all we need is fix the relation between p($\epsilon$) and the $\epsilon(r)$ that will give us the required equation of the state. 
\paragraph{ }
As we said before the electrons are the first to degenerate due to their smaller ,so the white dwarfs can be considered as ideal fermi gas of degenerate electrons but for it to be electrically the a sufficient amount of protons have to be added ,and the protons will from a nuclei together with neutrons but the nuclei don't contribute much to energy density (except the rest mass energy part) due to their low momentum compared to their rest mass energy.
\newpage
\paragraph{ }
As the white dwarf is being considered as ideal fermi gas of degenerate electrons thus they must follow the F-D statistic,and the distribution function of such system is given by;
\begin{center}
\begin{equation}
f = \frac{1}{exp\{(E-\mu)/k_B T\} +1}
\label{9}
\end{equation}
\end{center} 
\paragraph{ }
where E means the energy , and $\mu$ means chemical potential which basically means the change in the energy with per unit change in number density of particles, $k_B$ is the Boltzman constant and T being the temperature of the system
\paragraph{}
In kinetic theory the relation between the distribution function f and the number density $n_i = dN_i /V$ in phase space of a particular  particle say i is given by,
\begin{center}
\begin{equation}
\frac{dn_i}{d^3 k} = \frac{g}{\left(2\pi\hbar\right) ^3} f
\label{10}
\end{equation}
\end{center}
\paragraph{ }
$\left( 2\pi \hbar \right)^3$ is the unit volume of a cell in phase space and g is the the number of states allowed to a particle with in a given momentum, which is 2 for electrons,and $n_i$ denotes the number of i'th particle then the number density $n_i$ is given by ;
\begin{center}
\begin{equation}
n_i = \int dn_i = \int \frac{g}{\left( 2\pi\hbar \right)^3} f d^3 k
\label{11}
\end{equation}
\end{center} 
\paragraph{ }
In case of white dwarf the fusion of fuel have been exhausted to slowly it will radiate all its heat away we can take T $\rightarrow$ 0 as a very crude case in that situation we can write $\frac{\mu}{k_B T} \rightarrow \infty $ thus in such case a particle to be added to a system the particle must have energy E $\leq E_F$. So we can write the distribution function in such low temperature more like a step function given as;
\begin{center}
\begin{equation}
f(E) =  1   for   E\leq E_F 
    = 0  for   E\leq E_F
\label{12}
\end{equation}
\end{center}
\newpage
\paragraph{ }
Now as we considering a simpler model of  white dwarf ,we tend to ignore the coloumbic interaction between the electrons themselves and between the electrons and protons; then we can write in a simpler way from \ref{12} the number density of electrons as follows
\begin{center}
\begin{equation}
n_e = \int_0 ^{\infty} \frac{2}{\left(2\pi\hbar\right)^3} fd^3k 
    = \frac{8\pi}{(2\pi\hbar)^3}\int_0 ^{k_F} k^2dk
    = \frac{k_F^3}{3\pi ^2 \hbar ^3}
\label{13}
\end{equation}
\end{center}
\paragraph{ }
Here the we took f=1 from the \ref{12} and the here $d^3k = 4\pi k^2 dk$ as say the volume in phase space V=$4\pi k^3 /3$ then the elementary volume in phase space is given by $d^3 k = dV = 4\pi k^2 dk$
\paragraph{ }
Now for a white dwarf to be electrically neutral there have to be 1 proton per electron to confirm its charge neutrality,now as we are trying to consider a very simple model to study white dwarf we presume that they are made of $C^{12}$ and $O^{16}$ mostly,then we have A/Z = 2.Due to their smaller mass their contribution in mass density $\rho $ is negligible and cause of their heavier masses proton and neutrons contribute mainly ,so while calculating the mass density of the white dwarf we mainly concentrate on the nucleon mass $m_N$,then the density of the white dwarf is given by ;
\begin{center}
\begin{equation}
\rho = n \cdot m_N \cdot \frac{A}{Z}
\label{14}
\end{equation}
\end{center}
\paragraph{ }
here n= $n_e$ electron number density and the term $\frac{A}{Z}$ comes from the fact that we are taking mass of nucleon yet putting number density of electron, as the number density of nucleon and number density of electron are not equal so this factor comes in. Therefore  from above equation we can write fermi momentum $k_F$ as
\begin{center}
\begin{equation}
k_F = \left( \frac{3\pi ^2 \rho}{m_N} \frac{Z}{A} \right)^{1/3}
\label{15}
\end{equation}
\end{center}
As said before the momentum of a nucleon the is negligible compared to the rest mass energy of nucleons at T$\rightarrow$ 0 the nucleon do not contribute much to the energy but in case of the electron gas the contribution to the pressure is dominant due to their large velocities.
\newpage
\paragraph{ }
Hence we can divide the energy of white dwarf into two parts one part comes from the rest mass energy of nucleons and another comes from the electrons where the contribution of nucleons dominate over the electrons , the complete expression for the energy density $\epsilon(r)$ is given as
\begin{center}
\begin{equation}
\epsilon = nm_N \frac{A}{Z}c^2 + \epsilon_{elec}(k_F)
\label{16}
\end{equation}
\end{center}
\paragraph{ }
here $\epsilon_{elec}(k_F)$ is the energy density term for electron with energy 
\begin{center}
\begin{equation}
E(k) = \left(k^2 c^2 + m_e ^2 c^4\right)^{1/2}
\label{17}
\end{equation}
\end{center}
\paragraph{}
We can write the $\epsilon_{elec}(k_F)$ as 
\begin{center}
\begin{eqnarray}
\epsilon_{elec}(k_F) &=& \frac{8\pi}{(2\pi \hbar)^3}\int_0 ^{k_F} E(k)k^2 dk \nonumber \\
	&=& \frac{8\pi}{(2\pi \hbar)^3}\int_0 ^{k_F} \left( k^2 c^2 + m_e ^2 c^4 \right)^{1/2}dk 
\label{18} 
\end{eqnarray}
\end{center}
\paragraph{ }
say  $ u = \frac{k}{m_e c}$  ,then we have ;
\begin{center}
\begin{eqnarray*}
\left( k^2 c^2 + m_e ^2 c^4 \right)^{1/2}k^2 &=& m_e c^2 k^2 \left( \frac{k^2}{m_e ^2 c^2} +1 \right)^{1/2} \frac{m_e ^2 c^2}{m_e ^2 c^2}\\
	&=& \frac{k^2}{m_e ^2 c^2} \left( \frac{k^2}{m_e ^2 c^2} +1 \right)^{1/2} m_e ^3 c^4 \\
	&=& u^2 (u^2 +1)^{1/2} m_e ^3 c^4
\end{eqnarray*}
\end{center}
then the whole term $\left( k^2 c^2 + m_e ^2 c^4 \right)^{1/2}k^2 dk$ 
transform as $u^2 (u^2 +1)^{1/2} m_e ^3 c^4 \ast m_e c du$ the last term $m_e cdu$ comes from the fact that $du= \frac{dk}{m_e c} $
\newpage
\paragraph{ }
Hence the equation \ref{18} becomes as 
\begin{center}
\begin{eqnarray}
\epsilon_{elec}(k_F) &=& \frac{8\pi}{(2\pi \hbar)^3} \int_0 ^{k_f /m_e c}m_e ^3 c^4 u^2(u^2+1)^{1/2} m_e c du \nonumber \\
	&=& \frac{8\pi}{8\pi ^3 \hbar ^3}m_e ^4 c^5 \int_0 ^{k_f /m_e c}u^2 (u^2+1)^{1/2}du \nonumber \\
	&=& \frac{m_e^4 c^5}{\pi ^2 \hbar ^3} \int_0 ^{k_f /m_e c} u^2 (u^2+1)^{1/2}du \nonumber \\
	&=& \epsilon_0 \int_0 ^{k_f /m_e c} u^2 (u^2+1)^{1/2}du \nonumber 
\end{eqnarray}
\end{center}
\paragraph{ }
here $\epsilon_0$ means 
\begin{center}
\begin{equation}
\epsilon_0 = \frac{m_e ^4 c^5}{\pi ^2 \hbar ^3}
\label{19}
\end{equation}
\end{center}
\paragraph{ }
Now to solve the above integration we have to take a substitution $ u= \tan \theta$ then we will have $ du = \sec ^2 \theta d\theta$ and the term $u^2 (u^2+1)^{1/2}du = \tan ^2 \theta \sec \theta \sec ^2 \theta d\theta$
,for less complication in calculation lets forget about the limits for now later we will re-substitute u in place of $ \tan \theta $ after integration. After substitution we get the above integration as;
\begin{center}
\begin{eqnarray}
\int  u^2 (u^2+1)^{1/2}du &=& \int \tan^2 \theta \sec ^3 \theta d\theta \nonumber \\
	&=& \int (\sec ^2 \theta -1 )\sec ^3\theta d\theta \nonumber \\
	&=& \int (\sec^5 \theta  - \sec^3 \theta ) d\theta \nonumber \\
	&=& \int \sec^5 \theta d\theta - \int \sec^3 \theta d\theta \nonumber 
\end{eqnarray}
\end{center}
\paragraph{ }
say I = $\int \sec^5 \theta d\theta$ then we have 
\begin{center}
\begin{eqnarray}
I	&=& \int \sec^3 \theta \cdot \sec^2 \theta d\theta \nonumber \\
    &=& \tan \theta \sec ^3 \theta - \int \left(\frac{d}{d\theta} \sec ^3 \theta \right) \cdot \tan \theta d\theta \nonumber \\
    &=& \tan \theta \sec ^3 \theta - \int 3\sec^2 \theta \sec \theta \tan \theta \cdot \tan \theta d\theta \nonumber \\
    &=& \tan \theta \sec ^3 \theta - 3\int \sec^3 \theta \tan^2 \theta d\theta \nonumber \\
    &=& \tan \theta \sec ^3 \theta - 3\int \sec^3 \theta (\sec^2 \theta -1) d\theta \nonumber \\
    &=& \tan \theta \sec ^3 \theta - 3\int (\sec^5 \theta - \sec ^3\theta ) d\theta \nonumber
\end{eqnarray}
\end{center}
\begin{center}
\begin{eqnarray}
	I &=& \tan \theta \sec ^3 \theta -  3I + 3\int \sec^3 \theta d\theta \nonumber \\
	4I &=& \tan \theta \sec ^3 \theta + 3\int \sec^3 \theta d\theta \nonumber \\
	I &=& \frac{1}{4}\tan \theta \sec ^3 \theta + \frac{3}{4}\int \sec^3 \theta d\theta \nonumber
\end{eqnarray}
\end{center}
\paragraph{ }
Now putting this value in the place of $\int \sec^5 \theta d\theta $ in the equation	$\int u^2(u^2+1)^{1/2}du$ we get 
\begin{center}
\begin{eqnarray}
\int u^2(u^2+1)^{1/2}du &=& \int \sec^5 \theta d\theta - \int \sec^3 \theta d\theta  \nonumber \\
	&=& \frac{1}{4}\tan \theta \sec ^3 \theta + \frac{3}{4}\int \sec^3 \theta d\theta - \int \sec^3 \theta d\theta  \nonumber \\
	&=& \frac{1}{4}\tan \theta \sec ^3 \theta - \frac{1}{4} \int \sec^3 \theta d\theta  \nonumber
\end{eqnarray}
\end{center}
\paragraph{ }
Let's say $I_1 = \int \sec^3\theta d\theta$ then solving it we have 
\begin{center}
\begin{eqnarray}
I_1 &=& \int \sec ^3 \theta d\theta \nonumber \\
	&=& \int \sec^2 \theta \cdot \sec \theta d\theta \nonumber \\
	&=& \tan \theta \sec \theta - \int \left(\frac{d}{d\theta}\sec \theta \right) \cdot \tan \theta d\theta \nonumber \\
	&=& \tan \theta \sec \theta  -\int \sec \theta \tan \theta \cdot \tan \theta d\theta \nonumber \\
	&=& \tan \theta \sec \theta  -\int (\sec ^3 \theta - \sec \theta) d\theta \nonumber\\
	&=& \frac{1}{2}\tan \theta \sec \theta  +\frac{1}{2}\int\sec\theta d\theta \nonumber \\
	&=& \frac{1}{2}\tan \theta \sec \theta  +\frac{1}{2}ln| \sec \theta + \tan \theta| \nonumber
\end{eqnarray}
\end{center}
\paragraph{ }
now putting the whole thing together we get
\begin{center}
\begin{eqnarray}
\int u^2(u^2+1)^{1/2}du &=&  \frac{1}{4}\tan \theta \sec ^3 \theta - \frac{1}{4} \int \sec^3 \theta d\theta  \nonumber \\
	&=& \frac{1}{4}\tan \theta \sec ^3 \theta - \frac{1}{4}(\frac{1}{2}\tan \theta \sec \theta  +\frac{1}{2}ln| \sec \theta + \tan \theta|) \nonumber 
\end{eqnarray}
\end{center}
\paragraph{ }
For above intregration ,re-substituting u in place of $\tan \theta $ and $(u^2+1)^{1/2}$ in place of $\sec \theta $ then the above equation takes the form,
\begin{center}
\begin{eqnarray}
\int u^2(u^2+1)^{1/2}du &=&  \frac{1}{4} u (u^2+1)^{3/2} - \frac{1}{8}u(u^2+1)^{1/2} -\frac{1}{8}ln| u +(u^2+1)^{1/2}| \nonumber \\
	&=& \frac{1}{8}\left( 2u(u^2+1)^{3/2} -u(u^2+1)^{1/2} -ln| u +(u^2+1)^{1/2}| \right) \nonumber 
\end{eqnarray}
\end{center}	
\paragraph{ }
Now for limit part the lower limit is 0 and upper limit is $k_F/m_e c$ say $x=k_F/m_e c$ the after putting the limit we have the whole intregration as 
\begin{center}
\begin{eqnarray}
\epsilon_0\int_0 ^x u^2(u^2+1)^{1/2}du &=& \epsilon_0 \frac{1}{8}\left[ 2u(u^2+1)^{3/2} -u(u^2+1)^{1/2} -ln| u +(u^2+1)^{1/2}| \right]_0 ^x \nonumber \\
	&=& \frac{\epsilon_0}{8}\left[ u(u^2+1)^{1/2}(2u^2+2-1) -ln| u +(u^2+1)^{1/2}| \right]_0 ^x \nonumber \\
	&=& \frac{\epsilon_0}{8}\left[ x(x^2+1)^{1/2}(2x^2+1) -ln| x +(x^2+1)^{1/2}| \right] 
\label{20}
\end{eqnarray}
\end{center}
\newpage
\paragraph{ }
Here $\epsilon_0$ carries the dimension of energy density $(dyne/cm^2)$,now for the pressure of the system is given by 
\begin{center}
\begin{equation}
p =\frac{1}{3}\frac{8\pi}{(2\pi\hbar)^3}\int_0 ^{k_F} kvk^2 dk
\label{21}
\end{equation}
\end{center}
here the term $\frac{1}{3}$ comes for the isotropic distribution of momentum and velocity given by $v= kc^2/E$
\begin{center}
\begin{eqnarray}
p(k_F)&=& \frac{1}{3}\frac{8\pi}{(2\pi\hbar)^3}\int_0 ^{k_F} kvk^2 dk \nonumber \\
	&=& \frac{1}{3}\frac{8\pi}{(2\pi\hbar)^3}\int_0 ^{k_F} \frac{c^2k^2}{E(k)} dk \nonumber \\
	&=& \frac{1}{3}\frac{8\pi}{(2\pi\hbar)^3}\int_0 ^{k_F} (k^2 c^2 +m_e ^2 c^4)^{-1/2}c^2 k^4 dk \nonumber 
\end{eqnarray}
\end{center}
\paragraph{ }
Now like solving the integration \ref{18} here we will apply the same thing $u =k_F/m_ec$ and also aplying \ref{19} we get 
\begin{center}
\begin{eqnarray}
p(k_F) &=& \frac{\epsilon_0}{3} \int_0 ^{k_F/m_e c}u^4(u^2+1)^{-1/2}du  \nonumber
\end{eqnarray}
\end{center}
\paragraph{ }
Now like we did previously taking $u= \tan\theta$ and substituting it in above equation we get $u^4(u^2+1)^{-1/2}du = \left(\tan^4\theta/\sec \theta\right)\cdot\sec^2\theta d\theta$ then we get the integration in the form
\begin{center}
\begin{eqnarray}
p(k_F) &=& \frac{\epsilon_0}{3} \int u^4(u^2+1)^{-1/2}du  \nonumber \\
	&=& \frac{\epsilon_0}{3}\int \tan^4\theta\sec\theta d\theta \nonumber \\
	&=& \frac{\epsilon_0}{3}\int \left(  \sec^2 \theta -1 \right)\left(  \sec^2 \theta -1 \right) \sec \theta d \theta \nonumber \\
	&=&\frac{\epsilon_0}{3}\left(\int \sec^5\theta d\theta - 2\int \sec^3 \theta d \theta + \int \sec \theta d\theta \right) \nonumber
\end{eqnarray}
\end{center}
\newpage
\paragraph{ }
all the terms in the above integration is solved previously thus writing down the result from pervious we get 
\begin{center}
\begin{eqnarray}
p(k_F) &=& \frac{\epsilon_0}{3} \left( \frac{1}{4}\tan \theta \sec ^3 \theta + \frac{3}{4}\int \sec^3 \theta d\theta -2\int \sec^3\theta d\theta + \int \sec\theta d\theta \right) \nonumber \\
	&=& \frac{\epsilon_0}{3} \left( \frac{1}{4}\tan \theta \sec ^3 \theta -\frac{5}{4}\int\sec^3\theta d\theta +\int\sec\theta d\theta \right) \nonumber \\
	&=& \frac{\epsilon_0}{3} \left[ \frac{1}{4}\tan \theta \sec ^3 \theta  -\frac{5}{4}( \frac{1}{2}\tan \theta \sec \theta  +\frac{1}{2}ln| \sec \theta + \tan \theta| ) + \int \sec \theta d\theta \right] \nonumber \\
	&=& \frac{\epsilon_0}{3} \left[ \frac{1}{4}\tan \theta \sec ^3 \theta  -\frac{5}{8}\tan \theta \sec \theta - \frac{5}{8}ln| \sec \theta + \tan \theta| + ln| \sec \theta + \tan \theta| \right] \nonumber \\
	&=& \frac{\epsilon_0}{24} \left[ 2\tan \theta \sec ^3 \theta  -5\tan \theta \sec \theta  + 3ln| \sec \theta + \tan \theta| \right] \nonumber
\end{eqnarray}
\end{center}
we left the limists of the integration intentionally of less complication ,if we re-substitute $\tan\theta =u$ and $\sec\theta=(u^2+1)^{1/2}$ and putting the limits we would get;
\begin{center}
\begin{eqnarray}
p(k_F) &=& \frac{\epsilon_0}{24} \left[ 2u( u^2+1)^{3/2} -5 u( u^2+1)^{1/2} +3ln| u+ ( u^2+1)^{1/2}| \right]_0 ^{k_F/m_ec} \nonumber \\
	&=& \frac{\epsilon_0}{24} \left[ u( u^2+1)^{1/2}(2u^2 +2 -5) + 3ln| u+ ( u^2+1)^{1/2}| \right]_0 ^{k_F/m_ec} \nonumber \\
	&=& \frac{\epsilon_0}{24} \left[ ( x^2+1)^{1/2}(2x^3 -3x) + 3ln| x+ ( x^2+1)^{1/2}| \right] 
\label{22}
\end{eqnarray}
\end{center} 
The energy density is dominated by the mass density while in case of  pressure the electronic contribution dominate the nucleons ,now our goal was to find a equation in form $p p(\epsilon)$ to do that lets consider some extreme case: like for z>>1 and x<<1 such that in one case $k_F>>m_ec$ and for another case$k_F<<m_ec$ which basically means in one case the kinetic energy is much smaller than the rest mass energy "non-relativistic" case and for the another case momentum is larger or comparable to the rest mass "relativistic" case
\newpage  
\paragraph{ }
Now considering the non-relativistic case i.e. $x<<1$ which reduce down to $u= \frac{k_F}{m_ec} << 1$ and therefore $u^4(u^2+1)^{-1/2} \approx u^4\cdot 1$ ,then we can write in this limit as ,
\begin{center}
\begin{eqnarray}
p(k_F) &=& \frac{\epsilon_0}{3} \int_0 ^{k_F/m_ec} u^4(u^2+1)^{-1/2}du \nonumber \\
	&\approx & \frac{\epsilon_0}{3} \int_0 ^{k_F/m_ec} u^4 du \nonumber \\
	&=& \frac{\epsilon_0}{15} \left( \frac{k_F}{m_ec} \right)^5 \nonumber \\
	&=& \frac{\hbar ^2}{15\pi ^2 m_e}\left(\frac{3\pi^2 \rho Z}{m_NA} \right)^{5/3} 
\label{23}
\end{eqnarray}
\end{center}
 here we use the expression of  $\epsilon_0$ from \ref{19},with the consideration $\epsilon =\rho \cdot c^2$ then we write the approx $\textit{EQUATION OF STATE}$ in the non-relativistic limit as 
\begin{center}
\begin{equation}
p \approx K_{non-rel} \epsilon^{5/3}
\label{24}
\end{equation}
\end{center}
where the $K_{non-rel}$ i given by 
\begin{center}
\begin{equation}
K_{non-rel} = \frac{\hbar ^2}{15\pi ^2 m_e}\left(\frac{3\pi^2 Z}{m_N c^2 A} \right)^{5/3} 
\label{25}
\end{equation}
\end{center}
\paragraph{ }
Now for the relativistic case where $x>>1$ such that $u=\frac{k_F}{m_ec}>>1 $ then we write for $u^4(u^2+1)^{-1/2} \approx u^4\cdot \frac{1}{u} = u^3$ then we arrive at the relativistic \textit{RELATIVISTIC EQUATION OF STATE} given by
\begin{center}
\begin{eqnarray}
p(k_F) &=& \frac{\epsilon_0}{3} \int_0 ^{k_F/m_ec} u^4(u^2+1)^{-1/2}du \nonumber \\
	&\approx & \frac{\epsilon_0}{3} \int_0 ^{k_F/m_ec} u^4\cdot \frac{1}{u}du \nonumber \\
	&=&  \frac{\epsilon_0}{3\cdot 4} \left( \frac{k_F}{m_ec} \right)^4 \nonumber \\
	&=& \frac{\hbar ^2}{12\pi ^2 m_e}\left(\frac{3\pi^2 \rho Z}{m_NA} \right)^{4/3} 
\label{a}
\end{eqnarray}
\end{center}
\paragraph{ }
The relativistic EoS is given as  
\begin{center}
\begin{equation}
p(\epsilon) \approx  K_{rel}\epsilon ^{4/3}
\label{26}
\end{equation}
\end{center}
And the $K_{rel}$ is given as,
\begin{center}
\begin{equation}
K_{rel} = \frac{\hbar ^2}{12\pi ^2 m_e}\left(\frac{3\pi^2 Z}{m_N c^2 A} \right)^{5/3} 
\label{27}
\end{equation}
\end{center}
\paragraph{ }
Any equation of state that can be written in form of relation between pressure (p) and energy density($\epsilon$) like 
\begin{center}
\begin{equation}
p = K\epsilon^{\gamma}
\label{28}
\end{equation}
\end{center}
are called "polytropic" equation , and as we see the equation of white dwarf reduce down to polytropic equation in extreme cases like relativistic and non-relativistic equation and here $\gamma$ is some constant and have different values for different process like for non-relativistic limit value for it is  $\gamma = 5/3$ while for relativitic case we have $\gamma= 4/3 $ and using these in \ref{7} and \ref{8} for a given central energy density.
\subsection{CHANDRASEKHER MASS LIMIT}
\paragraph{}
Here we derive the mass limit for the white dwarf for  ploytropic  EoS founded by Chandrasekhar , we start with the equation \ref{7}and \ref{8} then from \ref{8} we can write m in terms of dp/dr then substitute it into the equation \ref{7}
\begin{center}
\begin{eqnarray}
m & = & -\frac{c^2 r^2}{G\epsilon} \frac{dp}{dr} \nonumber \\
\frac{d}{dr}\left(-\frac{c^2 r^2}{G\epsilon}\frac{dp}{dr} \right)& = & \frac{4\pi r^2 \epsilon}{c^2} \nonumber \\
\frac{1}{r^2} \frac{d}{dr} \left( \frac{r^2}{\rho} \frac{dp}{dr} \right) & = & -4\pi G \rho 
\label{30}
\end{eqnarray}
\end{center}
now as we know equation any equation of the type $ p = K\epsilon ^{\gamma}$ is known as polytropic process where $\gamma$ is parameter  which is written is $\gamma = 1 + \frac{1}{n}$ and n is given as polytropic index , then the polytropic equation of state can be written as $p=p(\rho)$ then we have,
\begin{center}
\begin{equation}
p = K\epsilon ^{\gamma} = K \rho ^{\gamma}c^{2\gamma}
\label{31}
\end{equation}
\end{center}
\newpage
\paragraph{ }
we can obtain $\rho(r)$ by solving \ref{30} by using the initial condition 
\begin{center}
\begin{eqnarray}
\rho(r=0) = \rho_0 \neq 0 \\
\label{32}
\frac{d\rho}{dr}|_{r=0} = 0
\label{33}
\end{eqnarray}
\end{center}
which can be done as m(r=0)=0 and \ref{8} ,now the equation \ref{30} can be transformed into a dimensionalessone by the following substitutions 
\begin{center}
\begin{eqnarray}
r&=&a\xi \nonumber \\
a & \equiv & \left( \frac{(n+1)K\rho_0 ^{(1-n)/n}}{4\pi G} \right) ^{1/2} c^{(n+1)/n} 
\label{34}
\end{eqnarray} 
\end{center}
and the mass density is given by 
\begin{center}
\begin{eqnarray}
\theta &=& \theta(r) \nonumber \\
\rho(r) &=&  \rho_0 \theta^n 
\label{35}
\end{eqnarray}
\end{center}
The quantities  $\theta \& \xi $ are dimensionless density and radius respectively and the 'a' is just a scale factor then we can make the  equation \ref{30} dimensionless using the $\theta \& \zeta$ and then we equation takes the form
\begin{center}
\begin{equation}
\frac{1}{\xi^2} \frac{d}{d\xi}\left( \xi^2 \frac{d\theta}{d\xi} \right) = -\theta^n 
\label{36}
\end{equation}
\end{center}
The above equation is called Lane-Emden equation for a given polytropic index n ,now for this new equation the boundary condition can derived from equation \ref{32} and \ref{33} and the fact is that $\rho(r=0)=\rho_0$ which is given by 
\begin{center}
\begin{equation}
\theta(r=0) = 1 
\label{37}
\end{equation}
\end{center}
while using the fact that $\frac{d\rho}{dr}|_{r=0} = 0 $ we can write the 
\begin{center}
\begin{equation}
\frac{d\theta}{dr} \vert_{r=0} =0
\label{38}
\end{equation}
\end{center}
\newpage
\paragraph{ }
With the boundary condition \ref{37} and \ref{38} one can numerically  integrate the equation \ref{36} we get the solution decrease with radius and have zero for $a\xi = \xi_1$ such that $\theta(\xi_1)  =0$ which basically means $\rho(r_1 = a\xi_1)=0$ therefore the radius of the star is given by R=$r_1=a\xi_1$ which can be writen as 
\begin{center}
\begin{equation}
R=\left( \frac{(n+1)K\rho_0 ^{(1-n)/n}}{4\pi G} \right) c^{(n+1)/n}\xi_1
\label{39}
\end{equation}
\end{center}
Now using the equation \ref{35} and the equation \ref{39} we get the mass of the star is given by
\begin{center}
\begin{eqnarray}
M &=& \int_0 ^R 4\pi r^2 \rho dr \nonumber \\
	&=& 4\pi a^3 \rho_0 \int_0 ^{\xi_1} \xi ^2 \theta ^n d\xi \nonumber   
\end{eqnarray}
\end{center}
now we know using the value for $\theta^n$ from equation \ref{36} then we have 
\begin{center}
\begin{eqnarray}
M &=& 4\pi c^{(3n+3)/n} \left( \frac{ (n+1)K}{4\pi G} \right)^{3/2} \rho_0 ^{(3-n)/2n} \int_0 ^{\xi_1} \xi^2 \cdot |\frac{1}{\xi^2} \frac{d}{d\xi} \left( \xi^2 \frac{d\theta}{d\xi}\right) | d\xi \nonumber \\
	&=& 4\pi c^{(3n+3)/n} \left( \frac{ (n+1)K}{4\pi G} \right)^{3/2} \rho_0 ^{(3-n)/2n} \xi_1 ^2|\theta{'}(\xi_1)| 
\label{40}
\end{eqnarray}
\end{center}
Using \ref{39} for $\rho_0$ we get 
\begin{center}
\begin{equation}
M = 4\pi c^{(2n+2)/(n-1)} \left( \frac{ (n+1)K}{4\pi G} \right)^{n/(n-1)} \xi_1 ^{(n-3)/(n-1)} \xi_1 ^2|\theta{'}(\xi_1)| R^{(3-n)/(1-n)}
\label{41}
\end{equation}
\end{center}
The high density limit such that in relativistic case with $\gamma$= 4/3 we have 
\begin{center}
\begin{eqnarray}
\gamma &=& 4/3 \nonumber  \\
n &=& 3 \nonumber \\
\xi_1 &=& 6.896 \nonumber \\
\xi_1 ^2 |\theta{'}| &=&  2.018
\end{eqnarray}
\end{center}
\newpage
\paragraph{ }
now for K we take \ref{28} and say $\eta = A/Z$ now using this in the equation \ref{41} we get the relativistic radius and mass 
\begin{center}
\begin{equation}
R = \frac{1}{2} (3\pi)^{1/2} (6.896)\left(\frac{\hbar^{3/2}}{c^{1/2}G^{1/2}m_e m_N \eta} \right)\left(\frac{\rho_{crit}}{\rho_0} \right)^{1/3}
\label{42}
\end{equation}
\end{center}
where the $\rho_{crit}$ is given by 
\begin{center}
\begin{equation}
\rho_{crit} = \frac{m_N\eta m_e^3c^3}{3\pi ^2 \hbar^3}
\label{43}
\end{equation}
\end{center}
and for the mass  we have 
\begin{center}
\begin{equation}
M= \frac{1}{2}(3\pi)^2(2.018)\left( \frac{\hbar c}{G} \right)^{3/2}\left(\frac{1}{m_N\eta}\right)
\label{44}
\end{equation}
\end{center}
now putting the value of constants we get 
\begin{center}
\begin{equation}
M = 1.4312 \left(\frac{2}{\eta}\right)^2 M_\odot
\label{45}
\end{equation}
\end{center}
where we took $m_N$ as the neutron mass for white dwarf  which mostly consist of $C^{12}$ we took the atomic mass unit which give the maximum mass of white dwarf which is 1.45 of solar masses which is the renowned chandrasekhar mass limit,which shows that in relativistic limit the mass does not depend on the radius R or the central density $\rho_0$ any more 
\subsection{POLYTROPIC EoS}
\paragraph{ }
We know that the polytropic equation of state is give by 
\begin{center}
$p = K\epsilon^\gamma $
\end{center}
the complication of solving \ref{7}and \ref{8} is to find a a way to get a certain pressure and corresponding to the energy density $\epsilon(r)$ using the polytropic equation in the \ref{8} and \ref{7}
\newpage
\paragraph{}
 we get 
\begin{center}
\begin{eqnarray}
\frac{dp(r)}{dr} &=& -\frac{R_0 p(r)^{1/\gamma} m(r)}{r^2 K^{1\gamma}} \\
\label{46}
\frac{d\tilde{m}(r)}{dr} &=& \frac{4\pi r^2}{M_\odot c^2}\left( \frac{p(r)}{K}\right)^{1/\gamma}
\label{47}
\end{eqnarray}
\end{center}
where $\tilde{m(r)}= \frac{m(r)}{M_\odot}$ denote the dimensionless mass an the $M_\odot$ denote the  solar mass and $R_0= \frac{GM_\odot}{c^2}$ denotes the Schwarzschild radius. The units of pressure p(r) and energy density $\epsilon(r)$ is $dyne/cm^2$  the mass is expressed in $M_\odot$ and the radius is km.
\paragraph{ }
Now we can distinguish between the relativistic and the non-relativistic case using the value of $\gamma =4/3$ and $\gamma =5/3$
. Choosing the central pressure $p_0 = 2.3302 \ast 10^22$ $dyne/cm^2$
 where the satisfies the condition $k_f << m_ec^2$  for non-relativistic case and solving it newtonian method gives us the curve shown in fig.2.2 for mass-radius and mass-pressure relations.
 \paragraph{ }
 For relativistic case the central pressure should be such that it must follow the conditon $k_F>> m_e c^2$  and we have $p_0 = 5.619 \ast 10^25$  and the mass -radius and pressure radius relation is given by fig.2.3
\paragraph{ }
Now as we chose the condition as $\eta =2 $ and use the relativistic polytrop the numerical computation exactly matchs the mass limit of white dwarf as calculated i.e. Chandrasekher mass limit the figure shows white dwarf with radius 5710 km and of mass 1.431$M_\odot$ which fits our calculation for initial high pressure very well 
\begin{center}
\begin{figure}
\includegraphics[scale=0.5]{non-rela}
\caption{ mass-radius and pressure-radius relation for non-relativistic}
\end{figure}
\end{center}
\begin{center}
\begin{figure}
\includegraphics[scale=0.5]{rela}
\caption{ the mass-radius and pressure-radius curve for relativistic case}
\end{figure}
\end{center}

\chapter{PURE NEUTRON STAR}
\subsection{Non-relativistic polytropic case}
Here we tart looking at the pure non-relativistic neutron stars which is described by the EoS of a fermi gas of neutrons,then we have from \ref{13} we can write the number density of neutrons as 
\begin{center}
\begin{equation}
n_n = \frac{k_n ^2}{3\pi ^2 \hbar ^3}
\label{3.1}
\end{equation}
\end{center}
where the $k_n$ denote the fermi momentum of the neutrons and the mass density $\rho_n$ can be given as 
\begin{center}
\begin{equation}
\rho_n = n_n \cdot m_n
\label{3.2}
\end{equation}
\end{center}
the from the analogy of previous caluclation of white dwarf as fermi gas of electrons (\ref{18} to \ref{22}) we get
\begin{center}
\begin{eqnarray}
\epsilon(x)&=& \frac{\epsilon_0}{8}\left[(2x^3+x)(1+x^2)^{1/2}+\sinh^{-1}(x)\right] \\
\label{3.3}
p(x) &=& \frac{\epsilon_0}{24}\left[ (2c^3-3x)(1+x^2)^{1/2}+3\sinh^{-1}(x)\right]
\label{3.4}
\end{eqnarray}
\end{center}
here 
\begin{center}
\begin{eqnarray}
x &=& \frac{k_n}{m_n c} \\
\label{3.5}
\epsilon_0 &=& \frac{m_n^4 c^5}{\pi^2\hbar ^3}
\label{3.6}
\end{eqnarray}
\end{center}
Therefore the \ref{3.3} and \ref{3.4} reduce down for the non-relativistic case is (like done in \ref{23})
\begin{center}
\begin{eqnarray}
\epsilon(x)&\approx& \rho c^2 = \frac{\epsilon_0 x^3}{3} \\
\label{3.7}
p(x) &\approx & \frac{\epsilon_0 x ^5}{15}
\label{3.8}
\end{eqnarray}
\end{center}
Hence the the EoS can be descrived by a polytrope:
\begin{center}
\begin{equation}
p(\epsilon)= K_{non-rela}\epsilon^{5/3}
\label{3.9}
\end{equation}
\end{center}
where the $K_{non-rela}$ is given by 
\begin{center}
\begin{equation}
K_{non-rela} = \frac{\hbar^2}{15\pi ^2 m_n}\left(\frac{3\pi^2 }{m_n c^2} \right)^{5/3}= 6.428 \cdot 10^{-26} \frac{cm^2}{erg^{2/3}}
\label{3.10}
\end{equation}
\end{center}
\subsection{Ultra-relativistic case}
For ultra-relativistic limit $x>>1$ gives use the energy density and for the pressure of nucleons as 
\begin{center}
\begin{eqnarray}
\epsilon(x) &=& \frac{\epsilon_0 x^4}{4}\\
\label{3.11}
p(x) &=& \frac{\epsilon_0 x^4}{12} \rightarrow p = \epsilon/3
\label{3.12}
\end{eqnarray}
\end{center}
\paragraph{ }
but here the the numeric integration of general relativistic corrected EoS faces some major problem as the condition for terminating the integration ($p\leq0$) can not be reached  , the pressure converges monotonically as a function of radius while the radius of the radius of the star approches infinity which in not feasible as the mass does not go to infinity because dm/dr also reaches zero asymptotically.This problem with relativistic case is that taking from the intial pressure($p_0$) to 0 through out the star is not necessarily very good aprroximation as the pressure have to pass the region where neutrons become non-relativistic
\subsection{Neutron stars with protons and electrons}
\paragraph{ }
As we know the neutron are unstable a neutron star will not thus consist of neutrons only but with the neutrons there would some fininte ammount of protons and electron in the dense matter ; the electron and proton are produced by $\beta-decay$
\begin{center}
\begin{equation}
n \rightarrow p+e^- + \bar{\nu_e}
\label{3.13}
\end{equation}
\end{center}
There is also possible reverse case wheere electron and neutron combining producing neutron and neutrino
\begin{center}
\begin{equation}
 p +e^- \rightarrow n+\nu_{e^{-}}
\label{3.14}
\end{equation}
\end{center}
Due to the fact the all the reaction is happening in the equilibrium condition for a cold neutron star then it must folow the relation of chemical potential  as 
\begin{center}
\begin{equation}
\mu_n = \mu_p+\mu_e
\label{3.15}
\end{equation}
\end{center} 
here we consider the chemical potential of the neutrinos as zero ($\mu_{\nu} =0 $)due to the fact that the neutrinos escape without and interaction and it is assumed that the neutron star is electrically neutral thus we can write
\begin{center}
\begin{equation}
n_p = n_e \Leftrightarrow k_p = k_e
\label{3.16}
\end{equation}
\end{center}
as the number density of the protons and electrons are same then we have the \ref{10} that their momentum must be same too and from the definition of the chemical potential($\mu_i = \dfrac{d\epsilon_i}{dn_i}$) we can write 
\begin{center}
\begin{equation}
\mu_i = \dfrac{d\epsilon_i}{dn_i}=\left(k_i^2 c^2+m_i^2c^4 \right)^{1/2}
\label{3.17}
\end{equation}
\end{center}
where $n_i$=n,p,e now using the equation \ref{3.15} and \ref{3.16} we get the following 
\begin{center}
\begin{equation}
\left(k_n^2 c^2+m_n^2c^4 \right)^{1/2}-\left(k_p^2 c^2+m_p^2c^4 \right)^{1/2}-\left(k_p^2 c^2+m_p^2c^4 \right)^{1/2} = 0
\label{3.18}
\end{equation}
\end{center}
the above equation can be solved by taking the fermi momentum of the proton$k_p$ as a function of the fermi momentum of the  neutron $k_n$
\begin{center}
\begin{equation}
k_p(k_n) = \dfrac{\left[(k_n^2c^2 +m_n^2c^4 - m_e^2c^4)^2 - 2m_p^2c^4(k_n^2c^2+m_n^2c^4 +m_e^2c^4) m_p^4c^8 \right]^{1/2}}{2c(k_n^2c^2 + m_n^2c^4)^{1/2}}
\label{3.19}
\end{equation}
\end{center}
\newpage
\paragraph{ }
The total energy and pressure is then sum of the neutron stars constituents such that the sum of energy and pressure electrons protons and neutron 
\begin{center}
\begin{eqnarray}
\epsilon_{tot} = \sum _{i=n,p,e} \epsilon_i \\
\label{3.20}
p_{tot} = \sum_{i=n,p,e} p_i 
\label{3.21}
\end{eqnarray}
\end{center}
where $\epsilon_i $ and the $p_i$ is given by 
\begin{center}
\begin{eqnarray} 
\epsilon_i(k_i) &=& \dfrac{8\pi}{(2\pi \hbar)^3}\int _0  ^ {k_i}(k^2c^2+m_i^2 c^4)^{1/2}k^2 dk \\
\label{3.22}
p_i(k_i) &=& \dfrac{1}{3}\dfrac{8\pi}{(2\pi \hbar)^3}\int_0  ^{k_i}(k^2c^2+m_i^2 c^4)^{-1/2}k^4 dk 
\label{3.23}
\end{eqnarray}
\end{center}
then for no neutron there should the case 
\begin{center}
\begin{equation}
k_p(0) = \dfrac{\left[( m_n^2c^4 - m_e^2c^4)^2 - 2m_p^2c^4(m_n^2c^4 +m_e^2c^4) +m_p^4c^8 \right]^{1/2}}{2c( m_n^2c^4)^{1/2}}=1.264\cdot 10^{-3}m_nc
\label{3.24}
\end{equation}
\end{center}
which in return implies that \ref{3.19} can only be used if the value of $k_p>1.264\cdot 10^{-3}m_nc$ which will happen only for the situation where the pressure $p>p_{crit}=3.038\cdot 10^{24}dyne/cm^2$ , below which there would be no neutron  present any more, so called "neutron star" matter will be only made of protons and electrons the fig. 3.1  hows the equation of states for free gas of proton electrons and neutron in $\beta-equilibrium $ in the form of $\epsilon(p)$. Around the critical pressure  notable kink appears in the equation of the state which can be identifies with the strong onset of neutrons appearing in matter with their corresponding additional contribution to the energy density
\begin{center}
\begin{figure}
\includegraphics[width=0.8 \textwidth]{p vs E neutron}
\caption{the energy density as function of pressure p of neutron stars including protons and electrons where the neutron starts to appear around the point $p=3.264\cdot 10^{24} dyne/cm^2$}
\end{figure}
\end{center}









\end{document}
